\documentclass[a4paper,12pt]{article}

% Pacotes básicos
\usepackage[utf8]{inputenc}
\usepackage[T1]{fontenc}
\usepackage{lmodern} % Fonte
\usepackage[brazil]{babel}
\usepackage{geometry}
\usepackage{setspace}
\usepackage{graphicx}
\usepackage{xcolor}
\usepackage{titlesec}
\usepackage{fancyhdr}
\usepackage{enumitem}
\usepackage{hyperref}
\usepackage{tcolorbox}
\usepackage{booktabs}
\usepackage{array}
\usepackage{caption}
\usepackage{multicol}
\usepackage{pdfpages}

% Margens
\geometry{
    left=2.5cm,
    right=2.5cm,
    top=2.5cm,
    bottom=2.5cm
}

% Cores
\definecolor{titulo}{HTML}{003366}
\definecolor{boxbg}{HTML}{F2F7FF}
\definecolor{boxborder}{HTML}{003366}

% Hyperlinks
\hypersetup{
    colorlinks=true,
    linkcolor=titulo,
    urlcolor=blue
}

% Cabeçalho e rodapé
\pagestyle{fancy}
\fancyhf{}
\fancyhead[L]{\textbf{Relatório Estratégico}}
\fancyhead[R]{UFPB – LEAD / A²D}
\fancyfoot[C]{\thepage}

% Estilo dos títulos
\titleformat{\section}
{\normalfont\Large\bfseries\color{titulo}}{\thesection}{1em}{}

\titleformat{\subsection}
{\normalfont\large\bfseries}{\thesubsection}{1em}{}

% Box padrão
\tcbset{
    mybox/.style={
        colback=boxbg,
        colframe=boxborder,
        boxrule=0.8pt,
        rounded corners,
        arc=5pt,
        left=10pt,
        right=10pt,
        top=10pt,
        bottom=10pt,
    }
}

% CAPA
\begin{titlepage}
    \centering

    \includegraphics[width=5cm]{logo_ufpb.png}\\[2cm]
    
    {\Huge \textbf{Relatório Executivo}}\\[0.5cm]
    {\Large Curso de Música – UFPB}\\[0.2cm]
    {\large Análise Estratégica Baseada em Evidências}\\[3cm]

    \vfill
    {\large \textbf{LEAD – Laboratório de Estudos e Análise de Dados}}\\
    {\large A²D – Núcleo de Análise de Dados e Desempenho}\\[0.5cm]
    {\large \today}

\end{titlepage}

\newpage

\noindent
O presente relatório foi desenvolvido com o objetivo de apoiar a liderança acadêmica em seu processo de tomada de decisão baseada em evidências.\\[0.3cm]

Os dados provêm do sistema SAEGO e correspondem à coorte 2015-2017 do curso de Música (período de referência: 2023). \\[1cm]

\begin{tcolorbox}[mybox]
\textbf{Objetivo geral:} apresentar uma análise estruturada baseada no Mapa Estratégico,
abrangendo estudantes, processos internos, pessoas e tecnologia, e recursos.
\end{tcolorbox}

\noindent
\textbf{Figura 1 – Mapa Estratégico}

\begin{center}
\fbox{\parbox[c][7cm][c]{14cm}{\centering Inserir imagem aqui}}
\end{center}

\newpage

% CORPO DO DOCUMENTO
\section*{ ESTUDANTES }
\subsection*{ Estudantes }



% Título da subseção
\subsubsection*{ Êxito profissional — Ocupação }

\begin{tcolorbox}[mybox]
\textbf{Fonte:} SAEGO \\[0.2cm]

\textbf{Dados}
\begin{itemize}

    \item \textbfOcupacao Percentual: 48,7\%

    \item \textbfTotal Egressos: 80

    \item \textbfEgressos Ocupados: 39

\end{itemize}
\end{tcolorbox}

\bigskip

\textbf{Análise} \\[0.2cm]
O índice de ocupação dos egressos, 48,7 \% (39 de 80), indica um nível moderado de inserção profissional, refletindo que menos da metade dos formados ingressou no mercado de trabalho.  
Tal desempenho sugere que a formação oferecida ainda apresenta lacunas para atender às demandas locais e regionais, exigindo um maior alinhamento entre o currículo acadêmico e as oportunidades de emprego disponíveis.

\bigskip

\textbf{Recomendações}
\begin{itemize}[leftmargin=1cm]

    \item Fortalecer parcerias estratégicas com empresas, órgãos públicos e organizações sociais locais para criar oportunidades de estágio, trainee e vagas de emprego diretas aos egressos.

    \item Estender e formalizar programas de extensão e projetos práticos que coloquem os estudantes em contextos reais de mercado, reforçando a aplicação dos conteúdos acadêmicos.

    \item Implementar um sistema de acompanhamento de carreira (career center) que monitore a empregabilidade dos egressos, identifique áreas de baixa inserção e informe ajustes curriculares baseados em demanda regional.

    \item Revisar o currículo com o apoio de representantes do setor produtivo, inserindo competências emergentes e projetos de inovação que atendam às exigências do mercado de trabalho local.

    \item Criar um programa de mentoria e networking com ex-alunos, facilitando a transferência de conhecimento, o compartilhamento de oportunidades e a construção de uma comunidade profissional robusta.

\end{itemize}

\bigskip



% Título da subseção
\subsubsection*{ Êxito acadêmico — Diplomação }

\begin{tcolorbox}[mybox]
\textbf{Fonte:} SAEGO \\[0.2cm]

\textbf{Dados}
\begin{itemize}

    \item \textbfDiplomacao Percentual: 22,4\%

    \item \textbfTempo Medio Conclusao Anos: 6.8

    \item \textbfEvasao Percentual Estimado: 41,3\%

\end{itemize}
\end{tcolorbox}

\bigskip

\textbf{Análise} \\[0.2cm]
A taxa de graduação de 22,4\% mostra um desempenho acadêmico abaixo da média, indicando dificuldades na retenção e conclusão dos cursos. O tempo médio de conclusão, de 6,8 anos, excede o padrão típico de 4–5 anos para graduação, sinalizando atrasos no percurso estudantil. A evasão estimada em 41,3 \% reflete um alto nível de desistência, exigindo atenção à trajetória dos estudantes.

\bigskip

\textbf{Recomendações}
\begin{itemize}[leftmargin=1cm]

    \item Implantar um sistema de alerta precoce que monitore presença, desempenho e engajamento dos estudantes, permitindo intervenções personalizadas de tutoria e acompanhamento acadêmico desde os primeiros semestres.

    \item Revisar a estrutura curricular para reduzir a carga horária por semestre, introduzir módulos intensivos ou créditos transferíveis e oferecer cursos online que possibilitem ritmo acelerado de conclusão, diminuindo o tempo médio de graduação.

    \item Fortalecer o serviço de orientação acadêmica e de carreira, vinculando mentores, oficinas de competências profissionais e planos de trajetória que alinhem expectativas de trabalho com a formação recebida, visando reduzir a evasão e aumentar a taxa de diplomação.

    \item Estabelecer centros de apoio estudantil que ofereçam aulas de reforço, treinamento em técnicas de estudo e suporte psicológico, criando um ambiente propício à permanência e ao sucesso acadêmico.

\end{itemize}

\bigskip



\section*{ TERCEIRA MISSÃO }
\subsection*{ Profissionais de Alto Nível }



% Título da subseção
\subsubsection*{ Overqualification — Profissionais de Alta Complexidade }

\begin{tcolorbox}[mybox]
\textbf{Fonte:} SAEGO \\[0.2cm]

\textbf{Dados}
\begin{itemize}

    \item \textbfOverqualification Percentual: 12,5\%

    \item \textbfProfissionais Alta Complexidade: 6

    \item \textbfTotal Egressos: 48

\end{itemize}
\end{tcolorbox}

\bigskip

\textbf{Análise} \\[0.2cm]
O percentual de 12,5\% (6 de 48 egressos) que ocupa cargos de alta complexidade indica que a formação atende a um segmento de profissionais de alto nível. Contudo, essa proporção sugere que apenas uma parte limitada da turma entra em posições que exigem esse grau de especialização, sinalizando um potencial de expansão na oferta de oportunidades qualificadas.

\bigskip

\textbf{Recomendações}
\begin{itemize}[leftmargin=1cm]

    \item Ampliar a grade curricular com cursos avançados, certificações e trilhas especializadas que preparem os egressos para cargos de alta complexidade, de forma a aumentar o percentual de 12,5 \% (6 de 48) de profissionais inseridos em posições exigentes.

    \item Fortalecer parcerias estratégicas com empresas, órgãos públicos e organizações que demandam profissionais de alto nível, criando programas de estágios supervisionados, projetos de pesquisa aplicada e oportunidades de trainee que conectem diretamente os estudantes a esses mercados.

    \item Implantar um programa de mentoria e networking, conectando ex-alunos já ocupando cargos de alta complexidade com alunos atuais, de modo a transferir conhecimentos práticos, orientações de carreira e ampliar a visibilidade de oportunidades.

    \item Desenvolver um painel de indicadores de desempenho (KPIs) alinhados ao Balanced Scorecard, monitorando a evolução do percentual de egressos em posições de alta complexidade e ajustando a formação conforme os resultados obtidos, garantindo que a estratégia acadêmica responda de forma ágil às demandas do mercado.

\end{itemize}

\bigskip



\end{document}